\documentclass[]{article}
\usepackage{amsmath, amssymb}

%opening
\title{Rental Bike Share Problem}
\author{}

\begin{document}

\maketitle

\section{Introduction}
 
RentalBike is a bike sharing company specialized in short-term bicycle rentals along various areas of a city, offering different categories of bicycles. Clients may rent bicycles for any duration and may deliver the rented bicycles in a place different from where the rental started. However, the delivery place must be in one of the areas currently operated by RentalBike. Since clients are able to change the bicycles’ original position, RentalBike must often redistribute the surplus of bicycles according to the expected demand of each area. 
The decision of how many bicycles will be redistributed from one area to another is based on expected profits for each possible volume of bicycles to be moved. Therefore, before starting the relocation of bicycles, it is necessary to optimize the volume of bicycles to be moved from one area to the others. 
The following sections describe the problem to be solved. Read the input descriptions and assumptions before doing the tasks. 


\section{Formulation Problem}


\subsection{Definition}

\begin{itemize}
	\item $I$ → Conjunto de categorias de bicicletas.
	\item $J$ → Conjunto de áreas de destino.
	\item $s_i$ → Quantidade de bicicletas da categoria $c$ disponíveis(excedentes) na área de origem.
	\item $c_i$ → Espaço ocupado por uma bicicleta da categoria $i$ no caminhão.
	\item $T$ → Capacidade total do caminhão.
	\item $p_{ij,k}$ → Lucro esperado por realocar a $k$-ésima bicicleta da categoria $i$ para a área $j$.
	\item $x_{i,j}$ → Variável de decisão representando a quantidade de bicicletas da categoria $i$ enviadas para a área $j$, onde $x_{i,j}$ é inteiro não negativo.
\end{itemize}

\subsection{Função Objetivo}

A função objetivo é dada pela maximização do lucro total da redistribuição das bicicletas:

\begin{equation}
\max \sum_{i \in I} \sum_{j \in J} \sum_{k=1}^{x_{i,j}} p_{ij,k}
\end{equation}

\subsection{Variáveis de decisão}
This section describes all decision variable. As quantidades de bicicletas enviadas para cada área devem ser inteiras e não negativas:

\begin{equation}
	x_{i,j} \in \mathbb{Z}^+, \quad \forall i \in I, \forall j \in J
\end{equation}

\subsection{Parameters}

\begin{itemize}
	\item $s_i$ → Quantidade de bicicletas da categoria $c$ disponíveis(excedentes) na área de origem.
	\item $c_i$ → Espaço ocupado por uma bicicleta da categoria $i$ no caminhão.
	\item $p_{ij,k}$ → Lucro esperado por realocar a $k$-ésima bicicleta da categoria $i$ para a área $j$.
	\item $x_{i,j}$ → Variável de decisão representando a quantidade de bicicletas da categoria $i$ enviadas para a área $j$, onde $x_{i,j}$ é inteiro não negativo.
\end{itemize}

\section{Restrições}

This section describes all the constraints of the model.

\subsection{Capacidade do caminhão}

O espaço total ocupado pelas bicicletas transportadas não pode exceder a capacidade do caminhão:

\begin{equation}
\sum_{i \in I} \sum_{j \in J} c_i x_{i,j} \leq T
\end{equation}

\subsection{Limitação de bicicletas disponíveis}

A quantidade de bicicletas enviadas de cada categoria não pode ultrapassar o estoque disponível:

\begin{equation}
\sum_{i \in I} x_{ij} \leq s_i, \quad \forall i \in I
\end{equation}




\end{document}
